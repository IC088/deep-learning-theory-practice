\documentclass[9pt,twocolumn]{article}

\usepackage[margin=0.8in,bottom=1.25in,columnsep=.4in]{geometry}
\usepackage{amsmath}
\usepackage{amssymb}
\usepackage{listings}
\usepackage{color}
\usepackage{cite}

\title{
	50.039 Theory and Practice of Deep Learning\\
	Theory Homework 1
}

\author{Joel Huang 1002530}
\date{\today}

\begin{document}
\maketitle

\section*{Drawing data from a distribution of $(x,y)$}
\subsection*{Problem definition}
\begin{itemize}
\item We need to draw pairs of data $x$ and binary labels $y=0$ or $y=1$.
\item There are two Gaussians with cluster index $c(x)=1$ and $c(x)=2$, each with a different probability of getting class $y=0$ and $y=1$.
\item $x$ follows the Gaussian distributions of the two clusters.
\end{itemize}

\subsection*{Formulation}
\begin{itemize}
\item $P(x,y)=P(y,x,c(x)=1)+P(y,x,c(x)=2)$.
\item $y$ is only dependent on cluster index $c(x)$. So $P(y\,|\,x,c(x))=P(y\,|\,c(x))$.
\item $P(x,c(x))$ can be decomposed into the product of the distribution of the data points given the cluster, and the probability of choosing the cluster:
\begin{equation*}
	P(x,c(x)=1) = f(x\,|\,c(x)=1)\,P(c(x)=1)
\end{equation*}
\begin{equation*}
	P(x,c(x)=2) = f(x\,|\,c(x)=2)\,P(c(x)=2)
\end{equation*}
\item $P(y=0)$ is the product of the probability of generating label $y=0$ given the cluster, and the probability of choosing the cluster:
\begin{equation*}
\begin{split}
	P(y=0) =&P(y=0\,|\,x,c(x)=1)\,P(x,c(x)=1) \\
	       &+P(y=0\,|\,x,c(x)=2)\,P(x,c(x)=2)
\end{split}
\end{equation*}
\item $P(y=1)$ can be expressed in terms of $P(y=0)$ because the labels are binary. Using the relationship $P(y=1\,|\,c(x)) = 1-P(y=0\,|\,c(x))$:
\begin{equation*}
\begin{split}
	P(y=1) = &(1-P(y=0\,|\,x,c(x)=1))\,P(x,c(x)=1) \\
	       & + (1-P(y=0\,|\,x,c(x)=2))\,P(x,c(x)=2)
\end{split}
\end{equation*}
\end{itemize}

\subsection*{Expressing $P(x,y)$}
\begin{equation*}
\begin{split}
P(x,y)&=P(y,x,c(x)=1)+P(y,x,c(x)=2)\\
      &=P(y\,|\,x,c(x)=1)\,P(x,c(x)=1)\\
	  &+P(y\,|\,x,c(x)=2)\,P(x,c(x)=2)\\
\end{split}
\end{equation*}
For $y=0$,
\begin{equation*}
\begin{split}
	P(x,y=0) = & P(y=0\,|\,x,c(x)=1)\,P(x,c(x)=1)\\
	  &+P(y=0\,|\,x,c(x)=2)\,P(x,c(x)=2)\\
	= & 0.2 \cdot f(x\,|\,c(x)=1)\,P(c(x)=1)\\
	& + 0.7 \cdot f(x\,|\,c(x)=2)\,P(c(x)=2)\\
		= & 0.2 \cdot 0.5 \cdot f(x\,|\,c(x)=1)\\
	& + 0.7 \cdot 0.5 \cdot f(x\,|\,c(x)=2)
\end{split}
\end{equation*}
Similarly for $y=1$,
\begin{equation*}
\begin{split}
	P(x,y=1)=&P(y=1\,|\,x,c(x)=1)\,P(x,c(x)=1)\\
	  &+P(y=1\,|\,x,c(x)=2)\,P(x,c(x)=2)\\
     =&(1-P(y=0\,|\,x,c(x)=1))\,P(x,c(x)=1)\\
     &+(1-P(y=0\,|\,x,c(x)=2))\,P(x,c(x)=2)\\
	= & 0.8 \cdot f(x\,|\,c(x)=1)\,P(c(x)=1)\\
	& + 0.3 \cdot f(x\,|\,c(x)=2)\,P(c(x)=2)\\
	= & 0.8 \cdot 0.5 \cdot f(x\,|\,c(x)=1)\\
	& + 0.3 \cdot 0.5 \cdot f(x\,|\,c(x)=2)
\end{split}
\end{equation*}
\end{document}